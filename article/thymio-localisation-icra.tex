% vim: fileencoding=utf-8:tw=0:noexpandtab:ts=4:sw=4
% -*- coding: utf-8 -*-

\documentclass[letterpaper, 10pt, conference]{ieeeconf}

\IEEEoverridecommandlockouts
\overrideIEEEmargins   

% packages
\usepackage[mathletters]{ucs}
\usepackage[utf8x]{inputenc}
\usepackage{graphicx}
\usepackage{microtype}
\usepackage{amsmath}
\usepackage{url}

% title and authors

\title{\LARGE \bf
Localization of inexpensive robots with low-bandwidth sensors
}

\author{Shiling Wang$^{1}$ and Francis Colas$^{2}$ and Ming Liu$^{3}$ and Stéphane Magnenat$^{4}$% <-this % stops a space
\thanks{*This work was not supported by any organization}% <-this % stops a space
\thanks{$^{1}$Shiling Wang is with KU Leuven TODO add department
        {\tt\small Shiling.Wang@esat.kuleuven.be}}%
\thanks{$^{2}$Fraucis Colas is with INRIA Nancy Grand Est
        {\tt\small francis.colas@inria.fr}}%
\thanks{$^{3}$Ming Liu is with UNDEFINED TODO
        {\tt\small ming.liu@undefined.nowhere}}%
\thanks{$^{4}$Stéphane Magnenat is with Mobots, LSRO, EPFL
        {\tt\small stephane@magnenat.net}}%
}

\begin{document}

\maketitle
\thispagestyle{empty}
\pagestyle{empty}

\begin{abstract}
Lorem ipsum...
\end{abstract}

\section{Introduction}

Why localisation is useful for cheap robot in educational context, why it is not trivial?

\section{Related work}

Markov and Monte Carlo Localization, why not SLAM?

Localization techniques for cheap mobile robots.

\section{Material}

\subsection{Thymio robot}

Thymio and Aseba~\cite{magnenat2012arso, aseba2011tmech, riedo2013arso}.

\subsection{Map}

\subsection{Data acquisition}

\section{Model}

\subsection{Variables}

The model uses the following variables:
\begin{itemize}
\item $X_{1:t}$ 2-D pose at times $1..t$.
The pose vector consists of $x,y$ coordinates and an angle $\theta$.
\item $Z_{1:t}$ observations at times $1..t$.
The observation consists of two sensors located at the bottom of the robot, measuring the intensity of the ground.
\item $U_{1:t}$ odometry at times $1..t$.
The odometry consists of the left and right wheel speeds.
\end{itemize}

\subsection{Joint probability}

The joint probability is:
\begin{equation}
\begin{split}
& p(X_{1:t}, Z_{1:t}, U_{1:t}) = \\
& p(Z_t|X_t) p(X_t|X_{t-1}, U_{t}) p(U_t) p(X_{1:t-1}, Z_{1:t-1}, U_{1:t-1})
\end{split}
\end{equation}

\subsection{Question}

We want to estimate the pose $X_t$ at time $t$ knowing the observations $Z_{1:t}$ and the commands $U_{1:t}$, using a recursive filter:
\begin{equation}
\begin{split}
& p(X_t|Z_{1:t},U_{1:t}) = \frac{p(X_t,Z_t | Z_{1:t-1}, U_{1:t})}{p(Z_t|Z_{1:t-1}, U_{1:t})} \\
 &\propto p(Z_t | X_t) p(X_t | Z_{1:t-1}, U_{1:t}) \\
 &\propto p(Z_t | X_t) \sum_{X_{t-1}} p(X_t, X_{t-1} | Z_{1:t-1}, U_{1:t} ) \\
 &\propto p(Z_t | X_t) \sum_{X_{t-1}} p(X_t|X_{t-1}, U_t) p(X_{t-1} | Z_{1:t-1}, U_{1:t-1})
\end{split}
\end{equation}
%This is a recursive filter to estimate $X_t$ using the previous estimation $X_{t-1}$, the odometry $U_t$ and the observation $Z_t$.

\subsection{Distributions}

\subsection{Markov implementation}

\subsection{Monte Carlo implementation}

\subsection{Theoretical analysis}
% TODO place at correct place
% TODO text could be adapted, some equations removed, etc. in case it's too long.
For the aim of localizing a robot in a given know space, we can get an idea of the time required to be localized.
For the Markov Localization approach, we can see that there is a given number of discrete cells.
The amount of information needed to unambiguously specify one among them all is:
\begin{displaymath}
	H_{loc} = \log_2(N_cells) = \log_2\left(\frac{L\times W\times N_{\theta}}{h^2}\right),
\end{displaymath}
with $N_cells$ the number of cells, $L$ and $W$ the length and width of the environment, $h$ the size of the cell, and $N_{\theta}$ the number of discretization steps of the angle $\theta$ of the robot.
In our example with a 1.5\,m$\times$1.5\,m environment discretized with cells of 1\,cm and 5° angle, the amount of information needed for the localization is around 20.6 bits.

A binary sensor ideally yields 1 bit of information per measurement.
However in practice, there is a loss in information due to the sensor noise, characterized above with the $p_{\mbox{correct}}$ probability of the sensor to be correct:
\begin{displaymath}
	H_{noise} = H_{\mbox{b}}(1 - p_{\mbox{correct}}),
\end{displaymath}
where $H_{\mbox{b}}$ is the binary entropy function: $H_{\mbox{b}}(p) = -p\log_2(p) - (1-p)\log_2(1-p)$.
With $p_{\mbox{correct}}=0.95$ the loss in information is around 0.29 bit per measurement.

In addition to the noise, we need to take into account that our sensor measurements are not completely independent.
For example, when not moving, we observe always the same place and thus cannot really gain additional information besides being sure of the color of the current pixel.
In a discretized world, we thus need to estimate the probability of having changed cell in order to observe something new, which depends on the distance travelled and the size of the cells.
This problem is equivalent to the Buffon-Laplace needle problem of finding the probability for a needle thrown randomly on a grid to actually intersect the grid.\footnote{\url{http://mathworld.wolfram.com/Buffon-LaplaceNeedleProblem.html}}
In our case, the probability of not changing cell is given by:
\begin{displaymath}
	p_{\mbox{same}} = \frac{4d h - d^2}{\pi h^2},
\end{displaymath}
with $d$ the distance travelled.

We can then compute the conditional entropy for two successive ideal measurements separated by $d$:
\begin{displaymath}
	H(O_t | O_{t-1}) = H(O_t, O_{t-1}) - H(O_{t-1}).
	% That's a really generic formula, but giving the 2x2 matrix or the expression with the logs is probably too much
\end{displaymath}
With a robot moving at around 3\,cm/s with a timestep of 0.3\,s with 3\,cm cells, we have $d=0.9$\,cm and the loss of information due to the redundancy of around 0.33 bit.

The computation is similar with a second sensor placed on the robot.
The probability that they see the same cell based on the distance between them is exactly the same.
On our small robots, the sensors are 2.2\,cm apart so their redundancy causes a loss of information of 0.041 bit.

If we combine all these effects, we can bound the information that our robot gathers at each timestep to at most 1.05 bit and a time to localize of at least 6\,s.
In order to have a faster localization, the greatest changes could be to move faster, in order to reduce redundancy in the successive measurements, or to have better sensors.
Setting the sensors apart also reduces the redundancy between their information but given our grid size, they are sufficiently separated.

\subsection{Dataset collection}

\subsection{Parameter estimation}

\section{Results}

plot full length, mean random1 + random2, ml, xy
plot full length, mean random1 + random2, ml, theta
plot full length, mean random1 + random2, mcl (50, 100, 200, 400), xy
plot full length, mean random1 + random2, mcl (50, 100, 200, 400), theta

say which converge and how

plot short length, mean 5 random1 + 5 random2, ml, xy
plot short length, mean 5 random1 + 5 random2, ml, theta
plot short length, mean 5 random1 + 5 random2, mcl, xy
plot short length, mean 5 random1 + 5 random2, mcl, theta

\subsection{Synthethic tests}

\subsection{Real-time with Thymio}

\section{Discussion}

What works, what does not, why?

Why sampling-based approach is not so easy, but propose how it could be done.

\section{Conclusion}

This paper provides several contributions to the state of the art of localization of robots with low-bandwidth sensors:
\begin{itemize}
\item an empirical evaluation of Markov- and Monte Carlo-based implementations;
\item a predictive model allowing designers to select and place sensors to achieve a desired localization performance;
\item a process to let end-users create and provide their own environments, such as drawings made by children.
\end{itemize}
Together, these allow absolute positioning of educational mobile robots in the 100 Euro price range.
This opens many educational opportunities, such as the study of geometry, puzzles based on the physical space, and staging the creation of stories in the context of language education.
These new possibilities are key elements for robots to answer the need for embodied computational thinking education.

\bibliographystyle{IEEEtran}
\bibliography{thymio-localisation-icra}

\end{document}



