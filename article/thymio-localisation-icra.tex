% vim: fileencoding=utf-8:tw=0:noexpandtab:ts=4:sw=4
% -*- coding: utf-8 -*-

\documentclass[letterpaper, 10pt, conference]{ieeeconf}

\IEEEoverridecommandlockouts
\overrideIEEEmargins   

% packages
\usepackage[mathletters]{ucs}
\usepackage[utf8x]{inputenc}
\usepackage{graphicx}
\usepackage{microtype}

% title and authors

\title{\LARGE \bf
Probabilistic localisation on low-cost robot
}

\author{Shiling Wang$^{1}$ and Francis Colas$^{2}$ and Ming Liu$^{3}$ and Stéphane Magnenat$^{4}$% <-this % stops a space
\thanks{*This work was not supported by any organization}% <-this % stops a space
\thanks{$^{1}$Shiling Wang is with KU Leuven TODO add department
        {\tt\small Shiling.Wang@esat.kuleuven.be}}%
\thanks{$^{2}$Fraucis Colas is with INRIA Nancy Grand Est
        {\tt\small francis.colas@inria.fr}}%
\thanks{$^{3}$Ming Liu is with UNDEFINED TODO
        {\tt\small ming.liu@undefined.nowhere}}%
\thanks{$^{4}$Stéphane Magnenat is with Mobots, LSRO, EPFL
        {\tt\small stephane@magnenat.net}}%
}

\begin{document}

\maketitle
\thispagestyle{empty}
\pagestyle{empty}

\begin{abstract}
Lorem ipsum...
\end{abstract}

\section{Introduction}

Why localisation is useful for cheap robot in educational context, why it is not trivial?

\section{Related work}

Thymio and Aseba~\cite{magnenat2012arso, aseba2011tmech, riedo2013arso}.

To read, sort, cite, and say something about relevant papers

\section{Material}

\subsection{Thymio robot}

\subsection{Model}

\section{Method}

\section{Results}

\subsection{Synthethic tests}

\subsection{Data collection with Thymio}

\subsection{Real-time with Thymio}

\section{Discussion}

What works, what does not, why?

Why sampling-based approach is not so easy, but propose how it could be done.

\section{Conclusion}

It's great, not perfect of course.

Highlights the contributions

\bibliographystyle{IEEEtran}
\bibliography{thymio-localisation-icra}

\end{document}



