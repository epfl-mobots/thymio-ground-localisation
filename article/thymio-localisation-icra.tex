% vim: fileencoding=utf-8:tw=0:noexpandtab:ts=4:sw=4
% -*- coding: utf-8 -*-

\documentclass[letterpaper, 10pt, conference]{ieeeconf}

\IEEEoverridecommandlockouts
\overrideIEEEmargins   

% packages
\usepackage[mathletters]{ucs}
\usepackage[utf8x]{inputenc}
\usepackage{graphicx}
\usepackage{microtype}
\usepackage{amsmath}

% title and authors

\title{\LARGE \bf
Localization of inexpensive robots with low-bandwidth sensors
}

\author{Shiling Wang$^{1}$ and Francis Colas$^{2}$ and Ming Liu$^{3}$ and Stéphane Magnenat$^{4}$% <-this % stops a space
\thanks{*This work was not supported by any organization}% <-this % stops a space
\thanks{$^{1}$Shiling Wang is with KU Leuven TODO add department
        {\tt\small Shiling.Wang@esat.kuleuven.be}}%
\thanks{$^{2}$Fraucis Colas is with INRIA Nancy Grand Est
        {\tt\small francis.colas@inria.fr}}%
\thanks{$^{3}$Ming Liu is with UNDEFINED TODO
        {\tt\small ming.liu@undefined.nowhere}}%
\thanks{$^{4}$Stéphane Magnenat is with Mobots, LSRO, EPFL
        {\tt\small stephane@magnenat.net}}%
}

\begin{document}

\maketitle
\thispagestyle{empty}
\pagestyle{empty}

\begin{abstract}
Lorem ipsum...
\end{abstract}

\section{Introduction}

Why localisation is useful for cheap robot in educational context, why it is not trivial?

\section{Related work}

Thymio and Aseba~\cite{magnenat2012arso, aseba2011tmech, riedo2013arso}.

To read, sort, cite, and say something about relevant papers

\section{Material}

\subsection{Thymio robot}

\subsection{Ground pattern}

\subsection{Model}

\subsubsection{Variables}

The model uses the following variables:
\begin{itemize}
\item $X_{1:t}$ 2-D pose at times $1..t$.
The pose vector consists of $x,y$ coordinates and an angle $\theta$.
\item $Z_{1:t}$ observations at times $1..t$.
The observation consists of two sensors located at the bottom of the robot, measuring the intensity of the ground.
\item $U_{1:t}$ odometry at times $1..t$.
The odometry consists of the left and right wheel speeds.
\end{itemize}

\subsubsection{Joint probability}

The joint probability is:
\begin{equation}
\begin{split}
& p(X_{1:t}, Z_{1:t}, U_{1:t}) = \\
& p(Z_t|X_t) p(X_t|X_{t-1}, U_{t}) p(U_t) p(X_{1:t-1}, Z_{1:t-1}, U_{1:t-1})
\end{split}
\end{equation}

\subsubsection{Question}

We want to estimate the pose $X_t$ at time $t$ knowing the observations $Z_{1:t}$ and the commands $U_{1:t}$, using a recursive filter:
\begin{equation}
\begin{split}
& p(X_t|Z_{1:t},U_{1:t}) = \frac{p(X_t,Z_t | Z_{1:t-1}, U_{1:t})}{p(Z_t|Z_{1:t-1}, U_{1:t})} \\
 &\propto p(Z_t | X_t) p(X_t | Z_{1:t-1}, U_{1:t}) \\
 &\propto p(Z_t | X_t) \sum_{X_{t-1}} p(X_t, X_{t-1} | Z_{1:t-1}, U_{1:t} ) \\
 &\propto p(Z_t | X_t) \sum_{X_{t-1}} p(X_t|X_{t-1}, U_t) p(X_{t-1} | Z_{1:t-1}, U_{1:t-1})
\end{split}
\end{equation}
%This is a recursive filter to estimate $X_t$ using the previous estimation $X_{t-1}$, the odometry $U_t$ and the observation $Z_t$.

\subsection{Markov implementation}

\subsection{Monte Carlo implementation}

\section{Method}

\subsection{Dataset collection}

\subsection{Parameter estimation}

\section{Results}

\subsection{Synthethic tests}

\subsection{Real-time with Thymio}

\section{Discussion}

What works, what does not, why?

Why sampling-based approach is not so easy, but propose how it could be done.

\section{Conclusion}

This paper provides several contributions to the state of the art of localization of robots with low-bandwidth sensors:
\begin{itemize}
\item an empirical evaluation of Markov- and Monte Carlo-based implementations;
\item a predictive model allowing designers to select and place sensors to achieve a desired localization performance;
\item a process to let end-users create and provide their own environments, such as drawings made by children.
\end{itemize}
Together, these allow absolute positioning of educational mobile robots in the 100 Euro price range.
This opens many educational opportunities, such as the study of geometry, puzzles based on the physical space, and staging the creation of stories in the context of language education.
These new possibilities are key elements for robots to answer the need for embodied computational thinking education.

\bibliographystyle{IEEEtran}
\bibliography{thymio-localisation-icra}

\end{document}



